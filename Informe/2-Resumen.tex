\thispagestyle{empty}
\begin{center}

	\textbf{\huge Resumen }\\[1cm] 

\end{center}

En el presente trabajo final se presenta el desarrollo del firmware del equipo Updown. La función principal de este dispositivo es la de subir y bajar una luminaria esférica a partir de las referencias de posición y velocidad recibidas por medio de una consola de luces. Como el equipo se utiliza en eventos de alta concurrencia de gente, también debe contar con mecanismos de detección de problemas de hardware que podrían causar algún daño a los espectadores, y a su vez evitar su rotura.

A la hora de comenzar el proyecto la mayoría del hardware y electrónica del mismo ya había sido desarrollado y probada. Por lo tanto, para que el Updown cumpla con su función se llevó a cabo el diseño, desarrollo e implementación del firmware necesario para que se pueda:

\begin{itemize}
	\item Manejar todas las entradas y salidas del equipo.
	\item Controlar la posición y velocidad de la luminaria.
	\item Detectar y actuar sobre posibles errores de hardware.
\end{itemize}

Adicionalmente, se concluyó con el desarrollo del sistema de direccionamiento del equipo (implementado mediante un Dipswitch), que es el único aspecto de hardware no resuelto del proyecto.

Para validar el funcionamiento del Updown se realizaron pruebas con varios equipos, analizando los distintos aspectos desarrollados durante este trabajo en una condición de uso similar a la que se someterán los mismos durante su utilización como producto.

Como conclusión se obtuvo que la mecánica de los equipos puede cambiar radicalmente el desarrollo del software, sumando dificultades extras y agrandando problemas que en la teória son fácilmente resueltos.


