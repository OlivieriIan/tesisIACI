\chapter{Desarrollo}
\thispagestyle{empty}

\section{Descripción del capítulo} \label{sec:\thesection}
En este capitulo se desarrollarán las soluciones propuestas en el capítulo de diseño. Esto implica detallar el proceso realizado, describir los cambios que se debieron hacer en caso de que el planteo inicial no haya funcionado, análisis de resultados, etc.

\section{Firmware del updown} \label{sec:\thesection}

\subsection{Librerías de bajo nivel}
Todas las librerías de bajo nivel fueron desarrolladas en base a la hoja de datos del Atmega328p.

\subsubsection{DigitalIO}
El desarrollo del módulo de entradas y salidas digitales (DigitalIO) está basado en el capítulo 18 de la hoja de datos del Atmega328p.
\subsubsection{ADC}
El desarrollo del módulo ADC está basado en el capítulo 28 de la hoja de datos del Atmega328p.

\subsubsection{PWM}
El desarrollo del módulo PWM está basado en el capítulo 20 de la hoja de datos del Atmega328p.

\subsubsection{EXINT}
El desarrollo del módulo de interrupciones externas (EXINT) está basado en el capítulo 17 y 18 de la hoja de datos del Atmega328p.

\subsubsection{UART}
El desarrollo del módulo UART está basado en el capítulo 24 de la hoja de datos del Atmega328p.

\subsubsection{SUART}
El desarrollo del módulo SUART está basado en el capítulo 19 de la hoja de datos del Atmega328p.

\subsubsection{Tick}
El desarrollo del módulo Tick está basado en el capítulo 22 de la hoja de datos del Atmega328p.

\subsection{Librerías de alto nivel}

\subsection{Funcion principal}

\section{Controlador} \label{sec:\thesection}

\subsection{Relación entre cuentas de encoder y distancia}
Como se mencionó en la sección \ref{sec:2.3}, subsección 3, 

\subsection{Determinación de la velocidad máxima}

\subsection{Obtención del período de muestreo}
Como se mencionó en la sección \ref{sec:2.3}, subsección 3, 

\subsection{Obtención del modelo de la planta}

\subsection{Obtención del controlador}

\subsection{Ajuste del controlador}


\section{Dipswitch} \label{sec:\thesection}


