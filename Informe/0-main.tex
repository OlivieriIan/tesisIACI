
\documentclass[12pt,spanish]{report}

%include spanish accents and tilde
\usepackage[T1]{fontenc}
\usepackage[utf8]{inputenc}

% ------------------- Colors --------------------------------------------------------------------------- %
%color package
\usepackage{xcolor}
% ==== hyper references colors ==== %
\definecolor{myLinkColor}{rgb}{0.6, 0.4, 0.1}
\definecolor{myUrlColor}{rgb}{0.1, 0.1, 0.7}

% ==== document control colors ==== %
% Each color specifies what is to be done with the text coloured with it
% command: \textcolor{color}{text paragraph}
\definecolor{commentary}{rgb}{0.5, 0.5, 0.5} %invisible mode: \iffalse ... \fi
\definecolor{FIXME}{rgb}{0.9, 0.1, 0.1} %changes have to be made here
\definecolor{TODO}{rgb}{0.2, 0.8, 0.4} %replace this with something else. probably an image

%Aplicar un color a un bloque de texto: \textcolor{FIXME}{ALGO}

% ==== other colors ==== %
\definecolor{colorUNQ}{RGB}{132, 0, 16}

% ------------------- margins control ------------------------------------------------------------------ %
%document margins control package
\usepackage{geometry}
%https://en.wikibooks.org/wiki/LaTeX/Page_Layout
\textheight = 620pt %Change the space occupied by the text. Default = 592pt, so text will occupy more space now
\textwidth  = 450pt %Change the space occupied by the text. Default = 390pt, so text will occupy more space now
\marginparwidth = 1pt % Default 35pt
\hoffset = 1pt

% ------------------- Footers and headers -------------------------------------------------------------- %
%total number of pages package 
\usepackage{lastpage} %\pageref{LastPage} to get the last page (* to remove reference)

%headers and footers package
\usepackage{fancyhdr} %\thispagestyle{fancy} to revert 'plain' page style from \chapter to fancy
\fancypagestyle{fancy_normalPage}
{
    \fancyhf{} % clear all header and footer fields
    %\fancyhead[L]{\rightmark}%\chaptername\ \thechapter}
    \fancyhead[R]{Página \thepage\ de \pageref*{LastPage}} %the * removes the reference
    \renewcommand\headrule %change header properties
    {
      \color{colorUNQ}
      \vspace{5pt}
      \hrule height 0.1mm width\headwidth
    }
    \renewcommand{\footrulewidth}{0mm} %erase footer line
}
% ------------------- Chapter related ------------------------------------------------------------------ %
%\renewcommand{\chaptername}{Capítulo} %Rename chapter name 
\usepackage{titlesec} %chapter names
\titleformat{\chapter}[display]
  {\normalfont\bfseries\centering}{}{0cm}{\Huge} %Delete the "Chapter n" header for each chapter
  \titlespacing*{\chapter}{0cm}{-2cm}{1cm} %resize title space {right}{above}{below}

% ------------------- Images --------------------------------------------------------------------------- %
%images package
\usepackage{graphicx}
%\graphicspath{ {resources/} }
\renewcommand{\figurename}{Figura} %name of the Figures

\iffalse %HOW TO USE FIGURES:
    \begin{figure}[!ht]
      \centering
        \includegraphics[width=\textwidth]{figname.png}
        \caption{description} %Caption appears below the image. place this line before \includegraphics to make it appear above it
        \label{fig:\thefigure}  %or \label{fig:aNumber}
    \end{figure}
    
    %referencing a figure
    \ref{fig:number} %if \thefigure is 1.1, then number = 1.1. if you used aNumber, number = aNumber
    \autoref{fig:number} %for the complete name hyperref (Figure: \thefigure)
\fi

% ------------------- Math package --------------------------------------------------------------------- %

\usepackage{amsmath}

\iffalse %EQUATIONS EXAMPLES
    %Normal equation
    \begin{equation}\label{eq:\theequation} %or \label{eq:aNumber} 
    \mbox{\Large\( % Change size of the containing box. Normal size: \normalsize
    \begin{split} %Align equations with & and separate with \\
        y_1 & = \sum_{i=0}^{\infty} a_i x^i \\ %\notag para sacar el label
        y_2 & = \sum_{i=\sigma}^{\Phi} a_i x^i
    \end{split}
    \)} %
    \end{equation}
    
    %referencing an equation
    \eqref{eq:number} %equation hyperref with parenthesis. available with \usepackage{hyperref}
    \ref{eq:number} %reference to the equation, without parenthesis
    \autoref{eq:number} %for the complete name hyperref (Equation:\theequation)
    
    %Aligned equation (with &)
    \begin{align}\label{eq:2}
        e^{i\pi} & = \cos(\pi) + i\sin(\pi) \notag \\  %To remove tag and move to the next line (\\)
            & = -1 .
    \end{align}
    
    %Piecewise equation
    \[
     fe(t) =
      \begin{cases} 
    
           1    &  0 \leq t \leq 5
           0    & e.o.c. \notag \\
      \end{cases}
    \]
    
    %Equations formatting: https://en.wikipedia.org/wiki/Help:Displaying_a_formula
    %Online equations editor: https://www.codecogs.com/latex/eqneditor.php
\fi

% ------------------- Tables example ------------------------------------------------------------------- %
\iffalse %This section does not include a user package. It's just an example
%https://en.wikibooks.org/wiki/LaTeX/Tables
\begin{table}[!ht]
  \begin{center}
    \begin{tabular}{| l c r |}
    \hline
    1 & 2 & 3 \\
    4 & 5 & 6 \\
    7 & 8 & 9 \\
    \hline
    \end{tabular}
  \end{center}
  \caption{A simple table}
\end{table}
\fi

% ------------------- Hyperreferences ------------------------------------------------------------------ %
%Hyperref packages
\usepackage{hyperref}
\hypersetup{
    colorlinks,
    citecolor=green,
    filecolor=black,
    linkcolor=colorUNQ,
    urlcolor=colorUNQ
}
%\urlstyle{same}

% ------------------- adding code/scripts -------------------------------------------------------------- %

\usepackage{listings}

\definecolor{matlabComment}{RGB}{34, 139, 34} 
\definecolor{matlabString}{RGB}{160, 32, 240} 
\definecolor{matlabNumber}{RGB}{255, 128, 102} 
\definecolor{matlabKeyword}{RGB}{0, 0, 255} 

\lstset{frame=tb,
  language=Matlab,
  aboveskip=3mm,
  belowskip=3mm,
  showstringspaces=false,
  columns=flexible,
  basicstyle={\small\ttfamily},
  numbers=none,
  %basicstyle=, %change the properties of the text
  identifierstyle=\color{black},
  numberstyle=\color{matlabNumber},
  keywordstyle=\color{matlabKeyword},
  commentstyle=\color{matlabComment},
  stringstyle=\color{matlabString},
  breaklines=true,
  breakatwhitespace=true,
  tabsize=3
}
% to inclode block of code: \lstinputlisting[language=language]{resources/sourceName.languageType}
% to encapsulate code: \begin{lstlisting}[frame=single] code... \end{lstlisting}

% ------------------- subscripts and superscripts fix -------------------------------------------------- %
%\usepackage{fixltx2e}
%subscript: x\textsubscript{y} or \(x_y\)
%superscript: x\textsuperscrip{y} or \(x_y\)
%Fix in headings: \texorpdfstring (from \usepackage{hyperref})

% ------------------- Apply stylization ---------------------------------------------------------------- %
%Remove paragraph indentation
\setlength{\parindent}{0pt}
\pagestyle{fancy_normalPage} %set pagestyle to my pagestyle (fancy_normalPage)
%Stylize new paragraphs. To use it explicitely: " \par " 
\setlength{\parindent}{0.3cm} % Paragraph indentation 
\setlength{\parskip}{0.3cm} % Space between paragraphs

% ------------------- Table of contents ---------------------------------------------------------------- %
\renewcommand{\contentsname}{Índice de contenidos}

\renewcommand{\listfigurename}{Índice de figuras}

% ------------------- Contadores de subsecciones ---------------------------------------------------------------- %
\setcounter{secnumdepth}{4} % seting level of numbering (default for "report" is 3). With ''-1'' you have non number also for chapters
\setcounter{tocdepth}{4} % if you want all the levels in your table of contents


% ------------------- Document ------------------------------------------------------------------------- %


\begin{document}

\begin{titlepage}
%https://en.wikibooks.org/wiki/LaTeX/Title_Creation
\includegraphics[width=\textwidth]{resources/0-UNQlogo.jpg}\\ %University logo

\begin{center}
	\text{\large Departamento de Ciencia y Tecnología}\\[0.2cm] % 
	\text{\large Ingeniería en Automatización y Control Industrial}\\[0.5cm] 
	\textsc{\LARGE \textcolor{colorUNQ}{Control automático del equipo Updown} }\\[1cm] 
	\textbf{\large Olivieri, Ian Paulo}\\ [1cm]
\end{center}	

\textbf{\normalsize Director:}\\
\text{\normalsize \hspace{3cm} Pernia, Eric Nicolás}\\

\textbf{\normalsize Co-director:}\\
	\text{\normalsize \hspace{3cm} x}\\

\textbf{\normalsize Jurado:}\\
	\text{\normalsize \hspace{3cm} Safar, Felix }\\
	\text{\normalsize \hspace{3cm} Juarez, José}\\
	\text{\normalsize \hspace{3cm} y}\\
	\text{\normalsize \hspace{3cm} z}\\[1cm]

	%\rule{\linewidth}{0.1mm} %Separation line
\begin{flushright}
	\text{\normalsize Presentación: Septiembre de 2017}\\
	\text{\normalsize Quilmes, Buenos Aires, Argentina.}\\ [0.5cm]
\end{flushright} 

\end{titlepage}

\thispagestyle{empty}
\begin{center}

	\textbf{\huge Resumen }\\[1cm] 

\end{center}



Resumen del proyecto (1 carilla)


% ==== Tabla de contenidos ==== %
\tableofcontents

% ==== índice de figuras ==== %
\listoffigures

\chapter{Introducción}
\thispagestyle{empty}


\section{Marco temático}
\subsection{Esculturas cinéticas}
\subsubsection{Definición}
Las esculturas cinéticas (kinetic sculpture en inglés) son estructuras tridimensionales en donde el movimiento es una parte fundamental del conjunto. Para lograr el efecto de movimiento en el espacio estos sistemas se construyen con partes móviles que pueden cambiar de posición ya sea naturalmente por acción del viento, como se ve en la figura \ref{fig:1.1}, o de manera forzada. \\
% https://www.youtube.com/watch?v=D2HF-1xjpP8

\begin{figure}[!ht]
	\centering
	\includegraphics[width=10cm,scale=1]{resources/1_1-kinSculp.png}
	\caption{ Ejemplo de escultura cinética movida por aire, por Anthony Howe. Fuente:  \href{https://www.youtube.com/watch?v=N-1LpikCSR4}{LINK al video} }
	\label{fig:\thefigure}
\end{figure}

\newpage
\subsubsection{Aplicaciones y estado actual del arte}
Al ser obras que caen dentro del campo artístico suelen presentarse en museos y utilizarse para fines decorativos ya sea en parques o eventos. Sin embargo, el nivel de ingeniería y diseño que algunas de ellas requieren las tornan un interesante desafío intelectual y creativo.\\


Las aplicaciones puntuales de estructuras cinéticas a las que se hará foco en este informe, debido a la naturaleza del proyecto final, son aquellas en donde el efecto espacial se logra a través del movimiento en el eje vertical de objetos esféricos mediante motores. \\

Un ejemplo de aplicación de estas características se puede ver en la figura \ref{fig:1.2}. Allí se muestra una escultura presentada en el Museo de BMW, en Munich, Alemania, en donde 714 esféras metálicas son coordinadas para formas figuras como olas, gotas, y hasta la silueta de un auto \\
\begin{figure}[!ht]
	\centering
	\includegraphics[width=15cm,scale=1]{resources/1_2-kinSculp.png}
	\caption{ Escultura cinética en el museo BMW. Fuente: \href{https://www.youtube.com/watch?v=HVhVClFMg6Y}{LINK al video} }
	\label{fig:\thefigure}
\end{figure}

Otro ejemplo de aplicación se puede ver en la figura \ref{fig:1.3}, en una obra presentada por la empresa Build Up en un centro comercial en Fukuoka, Japón. Allí se instalaron 1000 luminarias esféricas RGB dispuestas en una matriz de 25x40 para generar figuras tridimensionales como planos y gausseanas, entre otras. En este caso los efectos espaciales se logran coordinando el movimiento de cada esfera independientemente, cada una manejada por un equipo motorizado.
\begin{figure}[!ht]
	\centering
	\includegraphics[width=15cm,scale=1]{resources/1_3-kinSculp.png}
	\caption{ Escultura cinética por parte de Build Up. Fuente: \href{https://www.youtube.com/watch?v=ICixCazf6-k}{LINK al video} }
	\label{fig:\thefigure}
\end{figure}

%\newpage


\subsection{Sistemas de iluminación}
\subsubsection{Equipos de luces}
En cualquier espectáculo o evento la iluminación es una parte vital del show, y a medida que estos fueron evolucionando también lo hicieron los equipos de luces. Partiendo de aparatos fijos en donde solo se podía variar la intensidad de luz, se pueden conseguir hoy en día dispositivos complejos con decenas de parámetros controlables.\\
Un notable ejemplo es el \href{http://preworks.at/index.php/en/products/led-automated-luminairies/shapeshifter}{Shapeshifter}, figura \ref{fig:1.4}, que cuenta con 7 módulos leds que pueden ser manejados independientemente.

\begin{figure}[!ht]
	\centering
	\includegraphics[width=7cm,scale=1]{resources/1_4-shapeshifter.jpg}
	\caption{Shapeshifter, de High End Systems. Fuente: \href{https://www.youtube.com/watch?v=LIIE3zZscYY}{LINK al video} }
	\label{fig:\thefigure}
\end{figure}

En el caso del sistema visto en la figura \ref{fig:1.3}, los parámetros controlables de los equipos son la posición, velocidad y colores de cada esfera.

\subsubsection{Consolas de control de luminaria}
Para controlar los sistemas de luces es necesario utilizar unas consolas especiales. Estas se comunican con las luminarias utilizando el estándar \textbf{DMX} y le indican a cada equipo el valor de sus parámetros en todo momento. \\

La manera más común para generar un efectos es indicando la progresión de uno o más parámetros desde un tiempo inicial a uno final. Al cambio de los parámetros entre 2 instantes de tiempo se las llama \textit{cues}, o entradas, y cuyo conjunto forma los efectos. \\
Dentro de las consolas que hay en el mercado para este tipo de control de equipos se pueden destacar las \href{https://www.highend.com/products/consoles}{consolas hog 4} de High End Systems, como la que se muestra en la figura \ref{fig:1.5}\\

Otra manera generarlos es a partir de equipos y softwares, como el \href{https://www.madrix.com/}{Madrix}, que tienen la capacidad de convertir videos a variaciones de parámetros, lo cual lo hace especialmente útil cuando se quieren crear \href{https://www.youtube.com/watch?v=mdbl5ks7Nu0}{efectos lumínicos complejos}.


\begin{figure}[!ht]
	\centering
	\includegraphics[width=10cm,scale=1]{resources/1_5-consolaHOG.png}
	\caption{Consola Hog4, de High End Systems. Fuente: \href{https://www.highend.com/products/consoles}{LINK a la imágen}}
	\label{fig:\thefigure}
\end{figure}

 

\section{DMX}
\subsection{definición e historia}
DMX, de \textit{Digital MultipleX}, es un estándar de comunicación digital ámpliamente utilizado para el control de sistemas de iluminación. 

El estándar DMX512, donde 512 significa que se envían 512 piezas de información, fue creado por la \textit{United States Institute for Theatre Technology} (USITT) en 1986 y transformado en DMX512/1990 tras una revisión de la USITT. En 1998 la \textit{Entertainment Services and Technology Association} (ESTA) cuadró DMX dentro de los estándares ANSI, modificación que fue aprovada por el instituto (ANSI) en 2004. Finalmente, en 2008 DMX tuvo una nueva revisión y se llegó a la versión actual llamada "E1.11 – 2008, USITT DMX512-A", o simplemente DMX512-A. A pesar de esto, el nombre comúnmente conocido del estándar es simplemente DMX, aunque no es indistinto ya que hay diferencia de compatibilidad entre las diferentes versiones.


\subsection{Capa física}
\subsubsection{Cableado y conectores}
DMX empléa el estándar EIA-485 como capa física, por lo que emplea por lo menos 3 lineas; A, B y C, en donde A y B es por los datos son transmidos, y C es masa. Los conectores utilizados son los XLR, tanto de 5 como de 3 pines. Un ejemplo del cable se puede ver en la figura \ref{fig:1.6} \\

\begin{figure}[!ht]
	\centering
	\includegraphics[width=8cm,scale=1]{resources/1_6-cableDMX.jpg}
	\caption{Cable DMX con conector XLR5. Fuente: wikipedia}
	\label{fig:\thefigure}
\end{figure}

\subsubsection{Topología}
La red de DMX consiste en un maestro y varios esclavos, conectados con una topología de bus multidrop (MDB) con nodos conectados entre sí, lo que normalmente se denomina como topología \textit{daisy chain}. En otras palabras, todos los equipos a controlar tienen una entrada y una salida conectadas entre sí, de manera tal de que se puede conectar un equipo y apartir de este equipo conectar el siguiente, y así sucesivente, como se ve en la figura \ref{fig:1.7}. Esto permite que el conexionado sea simple y que la red pueda ser fácilmente extendida.

\begin{figure}[!ht]
	\centering
	\includegraphics[width=15cm,scale=1]{resources/1_7-topologiaDMX.png}
	\caption{Conexionado en una red DMX. Fuente: wikipedia}
	\label{fig:\thefigure}
\end{figure}

\subsubsection{Señal}
Al emplear el estándar EIA-485 la señal es de tipo diferencial, de una frecuencia de 250KHz.


\subsection{Capa de enlace de datos}
\textcolor{FIXME}{canales, addressamiento, forma de paquete paquetes, direccionalidad (uni)}\\
y paquetes de largo variable para la comunicación entre maestro y esclavo (que suele ser unidireccional).

\subsection{RDM}




\section{Updown}
\subsection{Definición}

\subsection{Blackout}
Black-out es una empresa productora y proveedora de tecnología, cuyo objetivo es generar contenido audiovisual para grandes eventos. Para cumplir con este objetivo la empresa dedica recursos al desarrollo de productos tecnológicos innovadores, como es el caso de los equipos destinados a generar las llamadas Esculturas cinéticas (Kinetic sculpture en inglés).\\

\subsection{Productos que compiten en el mercado}
El equipos más conocido empleado para estas aplicaciones es el \href{http://www.eastsunlite.com/p31.html}{OrbisFly}, de \href{https://www.kinetic-lights.com/}{Kinetic Lights}. Estos dispositivos de orígen Chino trabajan con una esfera RGB de peso estándar de 1Kg, aunque soporta un peso máximo de 2Kg, que puede descender hasta 9 metros. Dispone de un display LCD y botoneras para el testeo y configuración el equipo\\

\newpage
\section{Justificación del proyecto}
La empresa Black-out se encargó de comenzar el desarrollo de los Up-down; los equipos necesarios para crear los efectos de esculturas cinéticas. El inconveniente es que solo fue armada la parte mecánica del mismo, ya que a efectos prácticos el firmware que se utilizaba para manejar el conjunto era pobre.\\
Ante la necesidad que se tiene que el equipo controle la posición y velocidad de la carga, y que además verifique errores de hardware durante su uso para evitar accidentes, surgió la necesidad de buscar a alguien capacitado para mejorar la programación y terminar el proyecto.
Es entonces que se presentó la oportunidad de realizar un trabajo en la empresa con el objetivo de completar lo que falta del proyecto, hacer que este cumpla ciertas especificaciones técnicas, y crear documentación sobre el firmware y software de prueba de hardware para que más equipos de las mismas características puedan ser producidos. De esta manera, para completar el objetivo se debe:
\begin{itemize}
	\item Implementar un control automático de posición y velocidad de la carga movida por el equipo para lograr los efectos deseados.
	\item Permitir el cambio de los setpoints de posición y velocidad mediante comandos DMX enviados al equipo por una consola de luces, como puede ser una \href{https://www2.highend.com/products/controllers/Hog4Console.asp}{consola HOG4}
	\item Manejar errores y excepciones de hardware para lograr que el producto sea seguro, siendo que será instalado en eventos con un alto nivel de concurrencia.
\end{itemize}





\chapter{Diseño}
\thispagestyle{empty}
\section{Análisis de los requerimientos a resolver}
A continuación se analizan los requerimientos presentados en la sección \ref{sec:1.5} con el objetivo de determinar qué se planea hacer para cumplirlos.

\subsection{REQ-01}
\subsubsection{Análisis}
El microcontrolador utilizado en la placa de control es el Atmega328p. Este tiene capacidad de manejar todas las entradas y salidas del sistema ya que cuenta con periféricos de entradas y salidas digitales, entradas analógicas, comunicación serie (UART) y timers para temporización y generación de PWM. El único inconveniente que presenta es que es un microcontrolador de 8 bits sin optimizaciones para operaciones de punto flotante, lo que puede resultar un problema para ciertas partes del proyecto. \\
Verificado que el microcontrolador es viable para el proyecto se pasa a analizar cómo se manejarán los periféricos del mismo. Aquí las opciones son: utilizar librerías ya hechas o generar unas propias. \\
Dentro de las hechas la más recomendable es la de arduino, ya que cuenta con librerías para este microcontrolador y estas están altamente testeadas ya que son utilizadas por miles de personas y mantenidas por una enorme comunidad de desarrolladores de software. La contra de arduino es que ciertas librerías están mal optimizadas al utilizar operaciones de punto flotante, o tendrán que ser modificadas para que sean útiles para este proyecto en particular. Además, para garantizar el correcto funcionamiento del firmware lo óptimo sería verificar que todas las librerías utilizadas no empleen delays bloqueantes, que no realicen operaciones innecesarias, que no derrochen recursos, etc.\\
Debido a que el tiempo de desarrollo no es un factor limitante no es imperativo utilizar librerías ya hechas, por lo que se opta por crear librerías propias para manejar los periféricos.

\subsubsection{Diagráma de módulos}
El diagrama propuesto para el manejo de periféricos es:

\textcolor{FIXME}{PONER DIAGRAMA}


\subsection{REQ-02}
Para hacer que las cargas sigan una referencia de posición y velocidad 
\subsection{REQ-03}
\subsection{REQ-04}
\subsection{REQ-05}

\section{Hardware}


dipsw, relación encoder-distancia, velocidad maxima (para distintos pesos), relacion encoder motor/encoder disco.

\section{Firmware}
Diagrama de clases más que nada

\section{Controlador}
Modelo de control.

\chapter{Desarrollo}
\thispagestyle{empty}

\section{Descripción del capítulo} \label{sec:\thesection}
En este capitulo se desarrollarán las soluciones propuestas en el capítulo de diseño. Esto implica detallar el proceso realizado, describir los cambios que se debieron hacer en caso de que el planteo inicial no haya funcionado, análisis de resultados, etc. \\
Cada sección será redactada en el orden que se hicieron, debido a que unas partes dependen de otras para poder desarrollarse.

\section{Firmware del updown - Librerías de bajo nivel} \label{sec:\thesection}

\subsection{DigitalIO}
El desarrollo del módulo de entradas y salidas digitales (DigitalIO) está basado en el capítulo 18 de la hoja de datos del Atmega328p. 

\textcolor{FIXME}{FIJATE SI NO ES MEJOR EMPEZAR POR LO DE QUE EL ACCESO SE HACE CON 3 FUNCIONES, Y DE AHI DECIR QUÉ HACE CADA FUNCIÓN}

\subsubsection{Uso}
En microcontroladores de la marca Atmel, como es el de este proyecto, las entradas y salidas digitales de manejan mediante 3 registros: el DDRn (Data Direction Register n), PORTn (Port n Data Register) y PINn (Port n Input Pins Address). La "n" en los registros se refiere al registro específico a ser accedido, que en el caso de este microcontrolador puede ser B, C o D.

Para configurar un pin, que es una pata del microcontrolador, como entrada o salida digital primero se elige la dirección del mismo, o sea, si se utilizará como entrada O como salida digital. Para esto se utiliza el registro DDRn, en donde escribir un bit de este registro en 1 significa configurar el pin asociado a este bit como salida o Output, mientras que si se escribe en 0 significa configurar al pin como entrada o Input. \\
Si el pin fue configurado como salida se setea su valor mediante el registro PORTn. Un 1 en un bit de este registro significa poner en HIGH (5 Volts) la salida asociada a ese bit, mientras que un 0 es un estado LOW (0 Volts).\\
Si el pin fue configurado como entrada se lee su valor del registro PINn, que puede ser 0 o 1 (HIGH o LOW). A un pin configurado como entrada se le puede, además, habilitar una resistencia pull-up interna del microcontrolador, haciendo que por defecto el canal esté en 1. El pull-up se encuentra deshabilitado por defecto, pero puede ser habilitado escribiendo un 1 en el bit análogo del registro PORTn.\\
Por ejemplo, si se quiere configurar el bit 3 del puerto B como entrada con pull-up y el bit 6 del puerto D como entrada en lenguaje C, se tiene:

\begin{lstlisting}[style=CStyle]
	/* --- Configuracion bit 3 puerto B como Input - Pullup --- */
	DDRB &= ~(1 << 3); // Configurcion como entrada
	PORTB |= (1 << 3); // Habilitacion pullup
	
	valorBit3PuertoB = PINB & (1 << 3); // Ejemplo lectura del bit
	
	/* --- Configuracion bit 6 puerto D como Output --- */
	DDRD |= (1 << 6); // Configurcion como salida
	
	PORTD |= (1 << 6); // Ejemplo escritura de un 1 en el bit
\end{lstlisting}

\subsubsection{Implementación}
El acceso al módulo se realiza a través de 3 funciones: una de inicialización (init), una de lectura (write) y otra de escritura (read), presentadas en la figura \ref{fig:3.1}. \\
Para facilitar la lecto-escritura y configuración de los pines se mapearon los bits de los puertos B,C y D a números, como se muestra en la tabla \ref{table:3.1}. El criterio de enumeración fue basado en los pines del Arduino UNO.

\begin{table}[!ht]
	\begin{center}
		\begin{tabular}{|c|c|c|}
			\hline
			\textbf{Puerto} & \textbf{Bit} & \textbf{Numero mapeado} \\
			\hline \hline
			D & 0, 1, 2, 3, 4, 5, 6, 7 & 0, 1, 2, 3, 4, 5, 6, 7 \\
			\hline
			B & 0, 1, 2, 3, 4, 5 & 8, 9, 10, 11, 12, 13 \\
			\hline
			C & 0, 1, 2, 3, 4, 5 & 14, 15, 16, 17, 18, 19\\
			\hline
		\end{tabular}
	\end{center}
	\caption{Mapeo de bits de los puertos a número}
	\label{table:\thetable}
\end{table}

\begin{figure}[!ht]
	\centering
	\includegraphics[width=6cm,scale=1]{resources/3_1-moduloDigitalIO.png}
	\caption{Diagrama del módulo DigitalIO}
	\label{fig:\thefigure}
\end{figure}



\subsection{ADC}
El desarrollo del módulo ADC está basado en el capítulo 28 de la hoja de datos del Atmega328p.

\subsubsection{Uso}

\subsubsection{Implementación}

\begin{figure}[!ht]
	\centering
	\includegraphics[width=6cm,scale=1]{resources/3_2-moduloADC.png}
	\caption{Diagrama del módulo ADC}
	\label{fig:\thefigure}
\end{figure}

\subsection{PWM}
El desarrollo del módulo PWM está basado en el capítulo 20 de la hoja de datos del Atmega328p.

\subsubsection{Uso}

\subsubsection{Implementación}

\subsection{EXINT}
El desarrollo del módulo de interrupciones externas (EXINT) está basado en el capítulo 17 y 18 de la hoja de datos del Atmega328p.

\subsubsection{Uso}

\subsubsection{Implementación}

\subsection{UART}
El desarrollo del módulo UART está basado en el capítulo 24 de la hoja de datos del Atmega328p.

\subsubsection{Uso}

\subsubsection{Implementación}

\subsection{SUART}
El desarrollo del módulo SUART está basado en el capítulo 19 de la hoja de datos del Atmega328p.

\subsection{Tick}
El desarrollo del módulo Tick está basado en el capítulo 22 de la hoja de datos del Atmega328p.

\subsubsection{Uso}

\subsubsection{Implementación}


\section{Controlador} \label{sec:\thesection}

\subsection{Relación entre cuentas de encoder y distancia}
Como se mencionó en la sección \ref{sec:2.3}, subsección 3, 

\subsection{Determinación de la velocidad máxima}

\subsection{Obtención del período de muestreo}
Como se mencionó en la sección \ref{sec:2.3}, subsección 3, 

\subsection{Obtención del modelo de la planta}

\subsection{Obtención del controlador}

\subsection{Ajuste del controlador}


\section{Dipswitch} \label{sec:\thesection}

\section{Firmware del updown - Librerías de alto nivel} \label{sec:\thesection}

\section{Firmware del updown - Función principal} \label{sec:\thesection}



\input{7-Implementacion.tex}
\chapter{Conclusiones}
\thispagestyle{empty}

\section{Conclusiones} \label{sec:\thesection}
 
\section{Mejoras a futuro} \label{sec:\thesection}
\ref{sec:4.2}, subseccion 3. Tenes que decir que con una celda de carga o algo podes saber si se cae la carga (por corte de cadena o lo que sea) o si alguien se cuelga, haciendo al equipo muchísimo más seguro, y permitiendo cambiar de controlador según el peso directamente (aca podes desarrollar).

\section{Referencias utilizadas} \label{sec:\thesection}

\end{document}